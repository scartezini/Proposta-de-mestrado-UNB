\documentclass[12pt, a4paper]{article}

% Margins taken from a standard found on the internet.
%\usepackage[tmargin=1.9cm, bmargin=3.67cm, lmargin=1.9cm, rmargin=1.32cm]{geometry}
% LibreOffice's normal margin.
%\usepackage[tmargin=0.79in, bmargin=0.79in, lmargin=0.79in, rmargin=0.79in]{geometry}
% LibreOffice's narrow margin.
%\usepackage[tmargin=0.5in, bmargin=0.5in, lmargin=0.5in, rmargin=0.5in]{geometry}
%\usepackage[tmargin=0.6in, bmargin=0.6in, lmargin=0.6in, rmargin=0.6in]{geometry}
\usepackage[tmargin=0.60in, bmargin=0.60in, lmargin=0.79in, rmargin=0.79in]{geometry}

\usepackage{polyglossia}
\setdefaultlanguage{brazil} \setotherlanguage{english}

%% Tem que instalar essa fonte
\usepackage{fontspec}
\setmainfont[Ligatures=TeX]{Times New Roman}

\usepackage{setspace}
\onehalfspacing%

\usepackage{amsmath}
\numberwithin{table}{section}

\usepackage{titlesec}
\titlelabel{\thetitle.\quad}

\usepackage{booktabs}
\usepackage{multirow}
\usepackage{pifont}
% command to set spacing between lines in a booktabs table.
\newcommand{\ra}[1]{\renewcommand{\arraystretch}{#1}}

\usepackage{float}

% 'hyperref' is better being included last.
\usepackage{hyperref}

% ==============================================================================
% IDEIAS PARA ESCREVER A PROPOSTA
% ==============================================================================

\begin{document}
% =============================================================================
% Introdução
% =============================================================================
\section{Introdução}
\label{sec:introducao}
%- Problemática
%- Importância
%- Como eu vou resolver
%- Desafios

%% Porque fazer analise de sequencia de DNA, falar de descobrir novas
%% doencas e relacoes entre especies [Edans tese]
%% Explicar o conceito do algoritomo match missmatch gap
%% falar dos problemas uso de memoria O(mn) tempo de execução
%% Falar de algoritimos classicos SW (exato), FASTA BLAST (heuristico)
%% Limitacoes do SW
%% Otimizacoes do SW com openCL [Rucci]
%% Citar blockpruning [MASA]
%% propor blockpruning na fpga


O avanço do conhecimento na área da Biologia em especial na Genética permitiu que
os genomas de diversos organismos fossem sequenciados e armazenados em diversas 
bases de dados públicas. O processamento desse enorme volume de 
dados é muito interessante, pois pesquisadores podem: tirar conclusões sobre 
genes que causam doenças, comparar espécies do ponto de vista evolutivo, comparar 
o metabolismo entre diferentes especies, analisar mutações no genoma, entre 
outros. Grande parte dos avanços recentes se devem a Bioinformática. Nessa área se 
propõem novos algoritmos e o desenvolvimento de novas ferramentas para a analise
desses dados, gerando informações relevantes do ponto de vista 
biológico~\cite{Luscombe2001WhatIB}.

A comparação de duas ou mais sequencias biológicas é uma subárea fundamental
da Bioinformática. Esse tipo de processamento permite a busca de padrões 
entre sequências de aminoácidos e nucleotídeos. O algoritmo
Smith-Waternman (SW)~\cite{SMITH1981195} é um algoritmo clássico desta área 
de pesquisa, sendo frequentemente utilizado como base para 
pesquisas na área. Para  calcular o alinhamento local ótimo,
o algoritmo SW tem complexidade complexidade quadrática no tempo e espaço.

Pesquisas nas bases de dados de genomas requerem o calculo de alinhamento ótimo 
diversas vezes. Pela alta complexidade dos algoritmos exatos, o tempo de 
computação pode ser impraticável. Por essa razão surgiram algoritmos heurísticos, 
o FASTA~\cite{PMID:3162770} e o BLAST~\cite{ALTSCHUL1990403} são exemplos destes, 
para acelerar o processamento da comparação de duas sequencias, a custo de não 
garantir o resultado ótimo. 

Com o avanço tecnológico surgiram tecnologias capazes de viabilizar o processamento
dos algoritmos exatos. Utilizando-se de técnicas de programação paralela, e 
arquiteturas como por exemplo FPGAs (\textit{Field Programmable Gate Arrays}) e GPUs 
(\textit{Graphics Processing Units}), obtiveram-se bons tempos de execução no alinhamento
de genomas, garantindo o resultado ótimo. 

A partir de 1985, diversas pesquisas sugeriram a implementação do SW em FPGAs,
principalmente com foco no alinhamento de DNA. 
Em 2015 Rucci et al.~\cite{7345650} propuseram uma variação do algoritmo
Smith-Waterman utilizando OpenCL em FPGA.
Neste estudo obtiveram resultados expressivos tanto em termos de desempenho, 
tanto em termos de eficiência energética. 

Em 2016 Sandes et al.~\cite{DeO.Sandes:2016:MMA:2888415.2858656} propuseram
uma arquitetura multiplataforma para
alinhamento de sequencias (MASA) utilizando \textit{Block Pruning} (BP),
assim reduzindo consideravelmente o montante de dados a serem processados. 
Devido ao seus tempos de execução alcançados este trabalho possui muito bom
desempenho, tanto para GPU como para CPU.

% =============================================================================
% Justificativa
% =============================================================================
\section{Justificativa}
\label{sec:justificativa}

A arquitetura MASA~\cite{DeO.Sandes:2016:MMA:2888415.2858656}, permite a 
implementação do algoritmo SW para diferentes plataformas (CPU, GPU e acelerador 
Intel Xeon Phi). No entanto, não existe um implementação MASA para FPGA.
A implementação de um algoritmo clássico o SW em FPGA utilizando OpenCL para o 
alinhamento de sequencias, assim como a proposta de um novo método para este
tipo de processamento, como no caso do MASA, obtiveram o alto desempenho. 
A implementação do MASA utilizando BP em FPGA pode-se obter um ganho tanto 
de desempenho a vista do tempo de execução como uma maior eficiência energética
neste processamento, permitindo assim novos avanços nos estudos dos genomas.


% =============================================================================
% Objetivos
% =============================================================================
\section{Objetivos}
\label{sec:objetivos}
%% implementar um modulo do MASA em fpga
%% Ganho em eficiência energética
%% Ganho de desempenho

O objetivo principal deste trabalho é desenvolver uma extensão específica do 
MASA para a busca do alinhamento ótimo local entre sequencias de DNA, executando
em FPGA. Com o foco em obter uma melhora no desempenho do tempo de 
execução, com uma melhor eficiência energética. 

% =============================================================================
% Revisão da Literatura
% =============================================================================
\section{Revisão da Literatura}
\label{sec:revisao}

\subsection{Smith-Waterman}

Uma demanda muito comum da Bioinformática é a busca pelo alinhamento ótimo entre
duas sequencias, o algoritmo \textit{Needleman-Wunsch (NW)}~\cite{PMID:5420325}
obtém o alinhamento global 
ótimo duas sequências, permitindo que \textit{gaps} sejam inseridos para melhorar 
o alinhamento. Sendo que os alinhamentos globais incluem necessariamente 
todos os caracteres das sequências.
Uma situação mais comum é a busca do alinhamento local ótimo, onde se busca regiões
de alta similaridade entre as sequências. O \textit{Smith-Waterman (SW)}.
Proposto por Smith-Waterman em~\cite{SMITH1981195} esse algoritmo é baseado no NW,
porém adaptado para tratar o problema do alinhamento local.
O SW é executado em duas etapas. Primeiro, se utiliza de técnicas de programação 
dinâmica para calcular uma matriz, atribuindo diferentes valores para 
casos de \textit{match, mismatch} e \textit{gap}.
Após essa matriz ser preenchida, se realiza uma técnica chamada de \textit{traceback} 
onde encontra o alinhamento local ótimo.

\subsection {\textit{Smith-Waterman} com OpenCL em FPGA}

A grande limitação do SW é o custo computacional do algoritmo, tanto no tempo
quanto no espaço o SW tem complexidade quadrática o que o torna impraticável em
diversos cenários. Então surgiram diversas pesquisas para mitigar este problema
utilizando arquiteturas paralelas em CPU, GPU e FPGA. 
Rucci et al.~\cite{7345650} a eficiência de execução utilizando paralelismo a 
nível de \textit{thread} em uma FPGA Altera Stratix V. Para isso utilizaram da 
OpenCL, um \textit{framework} para implementação de programação paralela em
plataformas heterogêneas, suportando diversos \textit{hardwares}, como CPUs,
GPUs, FPGAs e outros. A OpenCL é baseada em um modelo \textit{host-device}, 
onde o \textit{host} é responsável pela gerencia de memória, transferência de dados e 
instanciação da execução nos \textit{devices}.
Rucci se utilizou dessa arquitetura para implementar o SW em 3 etapas, um pré-processamento
onde se adapta a sequencia de dados para a execução em FPGA, depois o estágio SW, onde é feito
o alinhamento das sequencias e por fim o estágio de ordenação que é feito o alinhamento 
das pontuações e ordenadas de forma decrescente. Sendo que o primeiro e terceiro estágios,
foram executados no \textit{host} e o segundo estagio na FPGA.
Com este trabalho Rucci chegou a conclusão que o paralelismo no nível de dados é 
crucial para obter boas taxas de desempenho ao custo de um aumento moderado no uso
de recursos. Além disso obteve um aumento no desempenho significativo em comparação
com implementações anteriores e um baixo custo energético.

\subsection {\textit{MASA: A Multiplatform Architecture for Sequence Aligners}}

O \textit{framework} MASA~\cite{DeO.Sandes:2016:MMA:2888415.2858656} 
dispõe uma infraestrutura flexível para o alinhamento de 
sequências em múltiplas plataformas. O MASA disponibiliza um 
código (desenvolvido em C/C++) que pode ser agregado a um desenvolvimento 
baseado em uma solução de processamento paralelo, permitindo a implementação
de soluções específicas. Para tanto o MASA é dividido em módulos de acordo 
com as funcionalidades existentes, algumas independentes de plataforma, como
o gerenciamento de dados e estágios, e funcionalidades dependente de plataforma, 
como processamento da matriz de programação dinâmica. Desta forma o MASA promove 
robustez e facilita o desenvolvimento de soluções em diferentes plataformas 
para o problema do alinhamento de sequencias, atualmente o MASA conta com 
implementações para CPU, GPU e acelerador Intel Xeon Phi.


% =============================================================================
% Metodologia
% =============================================================================
\section{Metodologia}
\label{sec:metodologia}

Para o desenvolvimento deste trabalho será utilizado o método experimental, 
com a implementação de uma extensão do MASA para uma execução eficiente 
da busca do alinhamento ótimo local entre duas sequencias. 
Comparando-o com implementações já existentes
além de implementações no MASA para outras plataformas (CPU, GPU e Intel Xeon Phi).

% =============================================================================
% Plano de Trabalho
% =============================================================================
\section{Plano de Trabalho}
\label{sec:plano}

Vinculado à proposta apresentada na Seção~\ref{sec:metodologia} este trabalho 
será concluído nas seguintes etapas:

\begin{enumerate}
	\item Revisão Sistemática da Literatura: Este item busca dar maior aprofundamento
		das especificidades relacionadas à realização deste trabalho.
	\item Cursar disciplinas do Programa de Pós-graduação em Informática da Universidade
		de Brasília
	\item Desenvolver o módulo proposto: Este item envolve avaliar, implementar e validar
		o sistema proposto.
	\item Elaboração de Artigos para Publicação em Revistas Científicas: A
			publicação é um dos instrumentos de validação da pesquisa desenvolvida
			perante a comunidade científica.
	\item Participação em Eventos de Ensino e Pesquisa: Objetiva a oportunidade de
			atualização e geração de conhecimento acerca de temáticas novas da área,
			além de ser uma oportunidade de envolvimento com pessoas que desenvolvem
			trabalhos na mesma linha de pesquisa.
	\item Confecção da Dissertação de Mestrado: A Dissertação constitui instrumento essencial
			para registrar e identificar de forma clara as contribuições científicas do
			trabalho executado.
\end{enumerate}

Os recursos necessários ao desenvolvimento do projeto envolvem a
disponibilização de laboratório com FPGAs, já disponíveis no
Programa de Pós-graduação em Informática da Universidade de Brasília.
% =============================================================================
% Cronograma
% =============================================================================
\section{Cronograma}
\label{sec:cronograma}

Pretende-se utilizar o primeiro semestre para uma Revisão Sistemática da Literatura,
o término do terceiro semestre para finalizar o desenvolvimento, e iniciar baterias 
de testes e fazer otimizações e a Defesa da Dissertação no ultimo semestre obedecendo o 
tempo previsto.

A Tabela~\ref{tab:cronograma} apresenta o cronograma de atividades prevista para o curso.

\begin{table}[H]
\label{tab:cronograma}
%\small
\centering
\caption{Cronograma de Atividades}
%\ra{1.2}
\begin{tabular}{@{}lllcllcllcll@{}}
\toprule
\multirow{2}{*}{Etapa / Semestre} & \multicolumn{2}{c}{$2020$} & 
                                  & \multicolumn{2}{c}{$2021$} \\

\cmidrule{2-3} \cmidrule{5-6} 

& $1^\circ$ & $2^\circ$ & 
& $1^\circ$ & $2^\circ$ \\

\midrule

Cursar Disciplinas do Mestrado          & \ding{117}    & \ding{117}    & 
                                        & \ding{117}    & 					    \\ 

Investigação Bibliográfica              & \ding{117}    & 					    & 
                                        &               &               \\

Desenvolvimento 						          	& \ding{117}  	& \ding{117}    & 
																				& \ding{117}    &               \\

Execução de Testes e Otimização         &               &               & 
																				& 					    & \ding{117}    \\ 


Escrita de Artigos para Periódicos      &               &               & 
                                        & \ding{117}    & \ding{117}    \\ 

Defesa da Dissertação                   &               &               & 
                                        &               & \ding{117}    \\
\bottomrule
\end{tabular}
\end{table}
\vspace*{-9mm}



% =============================================================================
% REFERÊNCIAS BIBLIOGRÁFICAS
% =============================================================================
%\clearpage
\renewcommand\refname{Referências Bibliográficas}
%\bibliographystyle{plain}
\bibliographystyle{abbrv}
\bibliography{refs}

\end{document}
