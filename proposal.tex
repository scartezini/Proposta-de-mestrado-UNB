\documentclass[12pt, a4paper]{article}

% Margins taken from a standard found on the internet.
%\usepackage[tmargin=1.9cm, bmargin=3.67cm, lmargin=1.9cm, rmargin=1.32cm]{geometry}
% LibreOffice's normal margin.
%\usepackage[tmargin=0.79in, bmargin=0.79in, lmargin=0.79in, rmargin=0.79in]{geometry}
% LibreOffice's narrow margin.
%\usepackage[tmargin=0.5in, bmargin=0.5in, lmargin=0.5in, rmargin=0.5in]{geometry}
%\usepackage[tmargin=0.6in, bmargin=0.6in, lmargin=0.6in, rmargin=0.6in]{geometry}
\usepackage[tmargin=0.60in, bmargin=0.60in, lmargin=0.79in, rmargin=0.79in]{geometry}

\usepackage{polyglossia}
\setdefaultlanguage{brazil} \setotherlanguage{english}

%% Tem que instalar essa fonte
\usepackage{fontspec}
\setmainfont[Ligatures=TeX]{Times New Roman}

\usepackage{setspace}
\onehalfspacing%

\usepackage{amsmath}
\numberwithin{table}{section}

\usepackage{titlesec}
\titlelabel{\thetitle.\quad}

\usepackage{booktabs}
\usepackage{multirow}
\usepackage{pifont}
% command to set spacing between lines in a booktabs table.
\newcommand{\ra}[1]{\renewcommand{\arraystretch}{#1}}

\usepackage{float}

% 'hyperref' is better being included last.
\usepackage{hyperref}

% ==============================================================================
% IDEIAS PARA ESCREVER A PROPOSTA
% ==============================================================================

\begin{document}
% =============================================================================
% Introdução
% =============================================================================
\section{Introdução}
\label{sec:introducao}
%- Problemática
%- Importância
%- Como eu vou resolver
%- Desafios

%% Porque fazer analise de sequencia de DNA, falar de descobrir novas
%% doencas e relacoes entre especies [Edans tese]
%% Explicar o conceito do algoritomo match missmatch gap
%% falar dos problemas uso de memoria O(mn) tempo de execução
%% Falar de algoritimos classicos SW (exato), FASTA BLAST (heuristico)
%% Limitacoes do SW
%% Otimizacoes do SW com openCL [Rucci]
%% Citar blockpruning [MASA]
%% propor blockpruning na fpga


O avanço do conhecimento na área da Biologia em especial na Genética permitiu que
os genomas de diversos organismos fosses sequenciados e armazenados em diversas 
bases de dados públicas~\cite{6545119}. O processamento desse enorme número de 
dados pode ser interessante, pois pesquisadores podem: tirar conclusões sobre 
genes que causam doenças, comparar espécies do ponto de vista evolutivo, comparar 
o metabolismo entre diferentes especies, analisar mutações no genoma, entre 
outros. Grande parte dos avanços se devem a Bioinformática. Nessa área se 
propõem novos algoritmos e o desenvolvimento de novas ferramentas para a analise
desses dados, gerando informações relevantes do ponto de vista 
biológico~\cite{Luscombe2001WhatIB}.

A comparação de duas ou mais sequencias biológicas e uma subárea fundamental
da Bioinformática, esse tipo de processamento permite a busca de padrões 
entre sequências de aminoácidos e nucleotídeos, quanto na busca de relações
filogenética entre organismos~\cite{Gollery2005BioinformaticsSA}. O algoritmo
Smith-Waternman (SW)~\cite{SMITH1981195} é um algoritmo clássico desta área 
de pesquisa, esse método é frequentemente utilizado como base para 
pesquisas na áreas, sendo um padrão comparativo para diferentes técnicas de 
alinhamento. Para  calcular pontuações ótimas as pontuações ótimas 
de alinhamento local, o algoritmo SW tem complexidade linear no espaço e 
complexidade quadrática no tempo.

Pesquisas nas bases de dados de genomas requerem o calculo de alinhamento ótimo 
diversas vezes. Pela a alta complexidade dos algoritmos exatos, o tempo de 
computação pode ser impraticável. Por essa rasão surgiram algoritmos heurísticos, 
o FASTA~\cite{PMID:3162770} e o BLAST~\cite{ALTSCHUL1990403} são exemplos destes, 
para acelerar o processamento da comparação de duas sequencias, a custo de não 
garantir o resultado ótimo. 

Com o avanço tecnológico surgiram tecnologias capazes de viabilizar o processamento
dos algoritmos exatos. Utilizando-se de técnicas de programação paralela, e 
arquiteturas como por exemplo FPGAs (\textit{Field Programmable Gate Arrays}) e GPUs 
(\textit{Graphics Processing Units}), obtiveram-se bons resultados no alinhamento
de genomas, garantindo o resultado ótimo. 

A partir de 1985, diversas pesquisas sugeriram a implementação do SW em FPGAs,
principalmente com foco no alinhamento de DNA. 
Em 2015 RUCCI, Enzo et al.~\cite{7345650} propuseram uma variação do algoritmo
Smith-Waterman utilizando OpenCL~\cite{Stone:2010:OPP:2220077.2220227} em FPGA,
neste estudo obtiveram resultados expressivos tanto em termos de desempenho, 
tanto em termos de eficiência energética. 

Em 2016 SANDES, Edans et al.~\cite{DeO.Sandes:2016:MMA:2888415.2858656} sob
orientação da prof. Alba de Melo propôs uma arquitetura multiplataforma para
alinhamento de sequencias (MASA) utilizando \textit{Block Pruning} (BP) 
---//TODO explicar o BP--- 
devido ao seus resultados este trabalho se encontra no estado da arte.

% =============================================================================
% Justificativa
% =============================================================================
\section{Justificativa}
\label{sec:justificativa}

A implementação de um algoritmo clássico o SW em FPGA utilizando OpenCL para o 
alinhamento de sequencias, assim como a proposta de um novo método para este
tipo de processamento, como no caso do MASA, obtiveram o alto desempenho. 
A implementação do MASA utilizando BP em FPGA pode-se obter um ganho tanto 
de desempenho a vista do tempo de execução como uma maior eficiência energética
neste processamento, permitindo assim novos avanços nos estudos dos genomas.


% =============================================================================
% Objetivos
% =============================================================================
\section{Objetivos}
\label{sec:objetivos}
%% implementar um modulo do MASA em fpga
%% Ganho em eficiência energética
%% Ganho de desempenho

% =============================================================================
% Revisão da Literatura
% =============================================================================
\section{Revisão da Literatura}
\label{sec:revisao}

%% MASA
%% BP


% =============================================================================
% Metodologia
% =============================================================================
\section{Metodologia}
\label{sec:metodologia}


% =============================================================================
% Plano de Trabalho
% =============================================================================
\section{Plano de Trabalho}
\label{sec:plano}

% =============================================================================
% Cronograma
% =============================================================================
\section{Cronograma}
\label{sec:cronograma}
\subsection{Disciplinas}
\subsection{Atividades de Pesquisa}



% =============================================================================
% REFERÊNCIAS BIBLIOGRÁFICAS
% =============================================================================
%\clearpage
\renewcommand\refname{Referências Bibliográficas}
%\bibliographystyle{plain}
\bibliographystyle{abbrv}
\bibliography{refs}

\end{document}
